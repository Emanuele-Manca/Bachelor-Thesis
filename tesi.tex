\documentclass[12pt,a4paper]{report}

% =============== PACKAGES =================
% \usepackage[T1]{fontenc}
\usepackage{template-tesi}
\usepackage{graphicx} % per le immagini
\usepackage{wrapfig} % metti le immagine in mezzo al testo
\usepackage{float} % metti le immagini dove vuoi 
\usepackage{hyperref} % per i link e riferimenti
\usepackage[italian]{babel} % utile per suddividere le parole quando si va a capo
\usepackage{geometry} % geometria della pagina
\usepackage{setspace}
\usepackage{listings} % mostrare codice
\usepackage{makecell}
\usepackage{nicefrac}
\usepackage{caption}
\usepackage{subcaption}

\usepackage{amsmath} % per ora utile per le frecce dei vettori in matematichese

\usepackage[backend=biber]{biblatex} % bibliografia
\usepackage{csquotes} % dipendenza di biblatex

% =============== GENERAL SETTINGS =========
% \geometry{a4paper,top=3cm,bottom=3cm,left=3cm,right=3cm}
\definecolor{backcolour}{rgb}{0.95,0.95,0.92}
\lstdefinestyle{mystyle}{
    backgroundcolor=\color{backcolour},
    % commentstyle=\color{codegreen},
    % keywordstyle=\color{magenta},
    % numberstyle=\tiny\color{codegray},
    % stringstyle=\color{codepurple},
    basicstyle=\ttfamily,
    breakatwhitespace=false,
    % breaklines=true,
    captionpos=t,
    keepspaces=true,
    % numbers=left,
    numbersep=5pt,
    % showspaces=false,
    % showstringspaces=false,
    showtabs=false,
    tabsize=4,
}
\hypersetup{
	bookmarksopen=true, 
	% TODO: switchare le due righe seguenti per la versione di stampa
	% hidelinks,
	colorlinks=true,
	citecolor=[RGB]{12, 133, 11},
	linkcolor=[RGB]{0, 32, 69},
	urlcolor=[RGB]{0, 73, 158},
}
\lstset{style=mystyle} % listing style
\pagestyle{headings}
\renewcommand\theadfont{\normalfont} % font per head delle tabelle
\renewcommand{\arraystretch}{1.3} % spazio interno alle celle di una tabella
\graphicspath{{images/}} % path delle immagini
% \renewcommand{\baselinestretch}{1.3} % intelinea
\addbibresource{bibliography.bib}

%================ FRONT PAGE ==============
\university{Università degli Studi di Milano}
\unilogo{images/unimi-logo}
\faculty{Facoltà di Scienze e Tecnologie}
\department{Dipartimento di Informatica\\Giovanni Degli Antoni}
\cdl{Corso di Laurea Triennale in Informatica\\Corso di Laurea}

\title{Strategie di\\Raceline Optimization\\in F1TENTH Autonomous Racing}
\author{Emanuele Manca}
\matricola{978785}

\typeofthesis{Elaborato Finale}

\relatore{Nicola Basilico}
\correlatore{Michele Antonazzi}

\academicyear{2023} 

% \tocintoctrue
% =========================================
\begin{document}
\makefrontpage
% \tableofcontents
% \beforepreface
\afterpreface
\newpage

% Chapter 1
% ======================================================================
% col: 20
\chapter{Introduzione}

Di seguito si descrivono le tecnologie usate e i concetti base che è necessario conoscere.

\section{F1TENTH}
F1TENTH è una community internazionale di ricercatori, ingegneri e appassionati di sistemi autonomi
che organizza competizioni di corsa di macchinine con la peculiare caratteristica
di essere \textit{un decimo} di quelle di F1, da cui il nome.
\begin{wrapfigure}{r}{0.3\textwidth}
	\centering
	\includegraphics[width=0.2\textwidth]{f1tenth-car.png}
	{\footnotesize Macchinina di F1TENTH}
\end{wrapfigure}
Oltre a questo, promuove la ricerca nell'ambito della guida autonoma
e altri campi tra cui reinforcement learning, sistemi di comunicazione e robotica;
offre, inoltre, corsi gratuiti online e una infrastuttura per imparare a costruire l'auto da corsa
e sviluppare il software necessario per farla gareggiare. \cite{ftenth-web}

La community è stata fondata all'Università della Pennsylvania nel 2016 ma ha iniziato rapidamente
collaborazioni con altre istituzioni e università in tutto il mondo~--~in Italia, alla data di stesura, solo
con l'Università degli Studi di Modena e Reggio Emilia.

\paragraph{I corsi}
Il percorso di apprendimento, che ha materiale per un semestre circa,
% TODO: \footnote{https://f1tenth.org/learn.html},
raggruppa le lezioni in moduli e li integra con altrettanti laboratori guidati
su cui poter applicare le nozioni e algoritmi imparati.

% TODO: da rivedere per non spaccare la lista in più pagine
% \begin{minipage}{\textwidth}
\noindent I moduli si raggruppano per tipologia di argomento trattato:
\begin{itemize}
	\item \textit{Modulo A} -- Introduzione a \hyperref[sec:ros]{ROS} e all'ambiente simulatore di \hyperref[par:gym]{F1TENTH~gym};
	\item \textit{Modulo B} -- Metodi reattivi e dinamiche del veicolo: \\
	      Descrive la modellazione delle dinamiche fisiche di un veicolo e presenta algoritmi di navigazione
	      reattivi come \textit{PID Control} e \textit{Wall following};
	\item \textit{Modulo C} -- Mapping \& Localization: SLAM e Particle Filter:\\
	      Introduzione alla stima dello stato, filtro di Bayes e varianti come Particle filter, uso di
	      algoritmi basati su filtri per la localizzazione del robot data una mappa, algoritmi localizzazione e
	      di modellazione dell'ambiente come SLAM;
	\item \textit{Modulo D} -- Planning \& Control: Pure Pursuit e RRT:\\
	      Descrizione dello stack di pianificazione e controllo di un veicolo autonomo, algortimo di
	      \textit{path tracking} Pure Pursuit basato sulla dinamica del veicolo, introduzione ai
	      \textit{local planner} e algoritmi come RRT, spline e clotoidi per descrivere un percorso,
	      State lattice planner, Graph based planner;
	\item \textit{Modulo E} -- Vision:\\
	      Introduzione a Classical Perception, differenze hardware, stima della distanza, visual SLAM,
	      percezione basata su ML, object detection;
	\item \textit{Modulo F} -- Argomenti speciali: \textit{Raceline Optimization}, MPC:\\
	      Ottimizzazione del percorso di gara, Model Predictive Control e Moral Decision Making;
	\item \textit{Modulo G} -- Applicazione delle tecnologie descritte su gara simulata;
\end{itemize}
% \end{minipage} \\

\noindent Ai fini dell'obiettivo di questa tesi si è preferito non approfondire il Modulo E,
in quanto si rendeva necessario l'uso fisico di hardware di cui non si aveva accesso,
ovvero l'auto e soprattutto sensori di visione come telecamere e LiDAr.

\paragraph{F1TENTH gym}
\label{par:gym}
% TODO: da rivedere
% https://docs.google.com/presentation/d/1zfzzjVTbXNIZ75BFtGEwQBJRHlY95VKkFQjJhYfznpI/edit#slide=id.g1cca33b1c19_0_3213
La community sviluppa e mantiene un simulatore open-source specifico per F1TENTH
% \footnote{https://github.com/f1tenth/f1tenth_gym_ros},
in modo da poter sviluppare e testare comodamente
senza l'uso di hardware specifico definito per la costruzione della macchinina.\\
Il simulatore definisce la dinamica del veicolo in modo tale da avere una simulazione
più reale possibile % Sim2Real problem
% e grazie a ROS e rviz (che sta per ROS Visualizer)

% TODO: parlare dei laboratori svolti?
% \subsection{Laboratori}


% ================== Autonomous Driving Pipeline =======================
\section{Autonomous Driving Pipeline}
Un progetto ben strutturato è più efficiente da mantenere e ha maggiori probabilità
di funzionare correttamente: ciò si applica anche in questo caso.\\
Un progetto complesso come quello dei veicoli autonomi deve essere suddiviso in sotto-problemi
più specifici e in modo tale che composti insieme risultino nella soluzione del problema.
\par

\noindent Il \emph{software} per i veicoli autonomi segue un ciclo di operazioni
in cui il prodotto di una fase è input della successiva; si indentificano in ordine tre fasi:
\begin{enumerate}
	\item \textbf{Perception} Attraverso sensori ottici e radar si \emph{percepisce} il mondo attorno a se,
	      ci si localizza nella mappa e si indentificano eventuali ostacoli o gareggianti;
	\item \textbf{Planning} Stando ai dati generati nella fase precedente,
	      si determinano quali saranno le \emph{mosse future} seguendo delle policy prescritte;
	\item \textbf{Control} Si generano dei comandi di angolo di sterzata e velocità per attuare le scelte
	      determinate nella fase precedente.
\end{enumerate}

L'input per la prima fase (\textit{Perception}) e l'ouput dell'ultima (\textit{Control})
è direttamente l'hardware: dunque nella prima sono i sensori ottici e radar, come citato precedentemente,
mentre nella seconda sono gli attuatori per i controllo della velocità e angolo delle ruote.\par
Il risultato che si vuole ottenere, quindi, è quello di una sequenza di comandi di sterzata e accelerazione
in un contesto, quello delle corse, che richiede una certa velocità di esecuzione per poter essere
il più reattivi possibile, dunque si tende ad eseguire il ciclo tra le 20 e 50 volte al secondo.\\
Una rappresentazione grafica della pipeline intera si trova alla figura \ref{fig:av-pipeline}\\
\begin{figure}[t]
	\centering
	\includegraphics[width=\textwidth]{AV-pipeline.png}
	\caption{Pipeline per i veicolo autonomi}
	\label{fig:av-pipeline}
\end{figure}

Analizziamo più nel dettaglio le fasi sopra citate. % TODO
\paragraph{Perception}

\paragraph{Planning}

\paragraph{Control}

% ================== ROS 2 ================================
\section{ROS2}
\label{sec:ros}
% cit: https://docs.ros.org/en/humble/Citations.html
% TODO: da continuare/rivedere
ROS è un acronimo che sta per \textbf{R}obot \textbf{O}perating \textbf{S}ystem e,
sebbene il nome possa trarre in inganno, non è un sistema operativo nel senso tradizionale,
bensì è più simile a un SDK (Software Development Kit).
% https://www.ros.org/blog/ecosystem/
Viene usato da tutta l'industria della robotica: dalla ricerca all'insegnamento, da progetti
di un gruppo di persone a progetti più importanti di grosse aziende.\\
ROS fornisce i \textit{building blocks} da usare per facilitare e velocizzare la realizzazione
del tuo progetto \dots
% TODO: parlare della community e pacchetti 
% TODO: parlare di ros2

ROS si presenta come un mediatore -- un \textbf{middleware} -- tra l'applicazione robot e l'hardware:
alla base fornisce un metodo di comunicazione tra le componenti dell'applicativo chiamati
\hyperref[ros:nodes]{\textbf{\textit{Nodi}}} secondo il pattern \textit{publisher/subscriber} di tipo anonimo.
L'insieme dei nodi e dei loro collegamenti formano il \textit{ROS~Graph}, il grafo dei nodi.
\bigskip

\noindent Di seguito si analizzano le componenti principali di cui ROS è composto.

\paragraph{Nodi}
\label{ros:nodes}
I nodi, \textit{Nodes} in inglese, sono l'unità computazionale che partecipa al ROS Graph;
idealmente, un nodo dovrebbe essere responsabile di un solo compito, logicamente separato
da altri altri, per esempio controllare le ruote, processare i dati grezzi derivati dalle misurazioni
dei laser o implementare un algoritmo come Pure Pursuit.
Un completo sistema robotico comprende, quindi, un insieme di nodi che lavorano congiuntamente, spesso
contenuti nello stesso eseguibile.

\paragraph{Topic}
\label{ros:topics}
I topic sono il principale strumento con cui i nodi di un ROS Graph si \textit{scambiano dati}.
I topic sono entità \textbf{\textit{fortemente tipizzate}} che implementano
un pattern \textbf{\textit{publisher/subscriber}} di tipo \textit{anonimo}, ciò significa
che i nodi conoscono una interfaccia comune con cui scambiarsi messaggi ben definiti.\\
\textit{Anonimo} significa che chi invia o riceve dei messagi da o verso un topic
non ha modo di identificare mittente o destinatario perchè, in questo contesto, non ha una grossa rilevanza,
sebbene esistano, comunque, dei modi per scoprirlo; questo ha il vantaggio di una grossa
flessibilità in termini di manuntenibilità.\\
I nodi possono \textit{pubblicare} e \textit{sottoscriversi} ad un numero arbitrario di topic
per inviare e ricevere dati da altri componenti del grafo: si ha quindi una relazione molti a molti.\\
I topic vengono identificati da una stringa univoca che ne rappresenta il nome.

\paragraph{Messaggi}

\paragraph{Parametri e Launch files}

%TODO: accenno a servizi e actions

%TODO: rviz

% Chapter 2

\chapter{Planning}
\label{chap:plan}

In questo capito si andaranno ad analizzare i compiti della seconda fase all'interno del ciclo della
\hyperref[sec:pipeline]{pipeline dei veicoli autonomi}, ovvero il planning. In questa fase viene inserito
l'argomento di studio di questa tesi: l'ottimizzazione della traiettoria globale.
Il planning, come suggerisce il nome, ha il compito di pianificare le mosse successive del robot affinchè
possa spostarsi in modo ottimale evitando eventuali ostacoli.

\bigskip
\noindent Gli algoritmi si differenziano per tecniche e obiettivi, in particolare si distinguono tre
macro-categorie che rispondono a domande diverse:
\begin{itemize}
	\item Mission/Global Planner: \textit{Qual è l'obiettivo generale del veicolo?} Per esempio trovare
	il percorso più breve, o quello più corto; 
	\item Behavioural Planner: \textit{Come dovrebbe comportarsi il veicolo in diverse situazioni?} Per
	esempio, in una gara con più concorrenti, quando superare un avversario, come e dove;
	\item Local Planner: \textit{Quali sono le traiettorie possibili dalla posizione attuale al goal?}
	Per esempio trovare la miglior traiettoria tale che rientri nella possibilità fisica del veicolo.
\end{itemize}
L'oggetto di questa tesi, l'ottimizzazione della traiettoria di corsa, rientra nella prima categoria.

% Tra i diversi algoritmi presenti in questo ambito si presenta una distinzione tra diverse
% caratteristiche.

\paragraph{\textit{Workspace} vs \textit{Configuration Space}} \ \\
Si distinguono due rappresentazioni degli ostacoli e dell'ambiente: il workspace rappresenta tutte le
azioni possibili a un robot, si pensi al volume occupato da tutti i possibili movimenti di un braccio
meccanico, mentre il conifuration space, detto anche C-space, rappresenta tutte le possibili
configurazioni di una data definizione di stato. In questo contesto, all'atto pratico, ciò si traduce che
nel secondo la rappresentazione dell'ambiente e degli ostacoli tiene già conto della dimensione del
robot e quindi quest'ultimo può essere considerato come un singolo punto, mentre ciò non accade nel
primo. La figura \ref{fig:work-vs-conf-space} ne mostra un esempio.

\begin{figure}[H]
	\begin{center}
		\includegraphics[width=0.95\textwidth]{work-vs-conf-space.png}
	\end{center}
	\caption{Differenze tra Workspace e C-Space}\label{fig:work-vs-conf-space}
\end{figure}

Nel contesto delle corse si preferisce l'uso del C-space per via della usa espressività nella descrizione
dello stato, che può non solo descrivere un piano o uno spazio tridimensionale, ma è possibile esprimere
ulteriori variabili come la velocità e l'orientamento: è così possibile esprimere ulteriori vincoli.

\bigskip
\noindent Di seguito si analizzano le tre macro-categorie sopra citate.

\section{Local Planner}
L'obiettivo principale del local planner è quello pianificare i movimenti del robot fino ad un dato
orizzonte finito evitando collisioni con l'ambiente ed eventuali avversari. Gli algoritmi si
differenziano per tre categorie principali di risoluzione:
\begin{enumerate}
	\item Applicando modifiche la raceline globale; 
	\item Generando diverse traiettorie vincolate alla dinamica del robot e scengliendo quella migliore;
	\item Campionando lo spazio libero e trovando un percorso attorno agli ostacoli.
\end{enumerate}
Nella prima categoria rientrano, generalmente, algoritmi che si basano su MPC che, sebbene sia un
algoritmo di controllo ottimale, può essere adattato e utilizzato per ricercare il percorso ottimo per
evitare un'ostacolo o per migliorare la traiettoria della raceline gloabale di riferimeto in base alla
posizione attuale del veicolo.

Nella seconda categoria ricadono algoritmi che calcolano, fino a un dato orizzonte temporale, lo stato
successivo del veicolo per diversi input di accelerazione e angolo delle ruote; questo operazione produce
diverse traiettorie, dinamicamente corrette per il robot, da cui si sceglie la migliore secondo una
funzione di costo. Altri algoritmi, come State Lattice, generano punti equidistanti nell'ambiente tra di
loro connessi da delle \hyperref[par:spline-def]{spline} che seguono la dinamica del veicolo, poi viene
ricercato il percorso migliore. Un esempio grafico è mostrato in figura \ref{fig:state-lattice}.

La terza categoria, data una occupancy grid e un punto di goal, algoritmi come PRM (Probabilistic Road
Map) e derivati come RRT (Rapidly-Exploring Random Tree) campionano lo spazio libero e generano un grafo
che connette i punti campionati per poi cercare il percorso che avvicina il robot al goal. RRT è stato un
caso di studio durante questa tesi durante un laboratorio del Modulo D. Un esempio di RRT si può trovare
alla figura \ref{fig:rviz-example}. Altre soluzioni usano i classici algoritmi di ricerca su grafo, come
A* e Dijkstra, sulla stessa occupancy grid; il grafo costruito a partire da essa esprime sui nodi le posizioni
nell'ambiente e gli archi i possibili input da quelle posizioni. Queste ultime sono soluzioni discrete,
molto semplici da implementare ma che non hanno la stessa espressività dei metodi continui e che non
hanno la possibilità di descrivere la dinamica del robot.

\begin{figure}[h]
	\begin{center}
		\includegraphics[width=0.5\textwidth]{state-lattice.png}
	\end{center}
	\caption{Esempio grafico dell'algoritmo State Lattice}\label{fig:state-lattice}
\end{figure}

\section{Behavioural Planner}
Il focus di questo planner è generalmente selezionare un peso appropriato a diversi obiettivi o combinare
il risultato del local planner con metodi della teoria del gioco così da pianificare delle mosse atte a
impedire il progresso degli avversari.

Nel primo caso, gli obiettivi rappresentano valori come il progresso sul circuito, la vicinanza con gli
ostacoli e contendenti, la deviazione dal percorso ottimale, la velocità massima; il costo totale di una
traiettoria viene quindi calcolato combinando secondo vari pesi gli obiettivi presi in considerazione,
viene quindi scelto la traiettoria col costo minore.

Nel secondo caso vengono sfruttate metodologie della teoria dei giochi per trovare l'azione migliore in
un ambiente con due o più giocatori. Il problema viene trasformato in gioco competitivo asincrono dove un
singolo giocatore può "muoversi" per volta. Questi approcci spesso includono nel cacolo della soluzione il
concetto di \textit{regret} per trovare la migliore risposta per vincere la gara.

\section{Global Planner}
Come citato più volte precedentemente, questa tipologia di planner è stato oggetto di studio di questa
tesi e quindi ha dedicato un capitolo, il numero \ref{chap:opt} (pag. \pageref{chap:opt}).

Il global planner ha una visione d'insieme del circuito ed è agnostico alla singola gara, perciò i suoi
obiettivi sono legati a proprietà della traiettoria (globale) da seguire; generalmente l'obiettivo
principale è eseguire il tracciato in meno tempo possibile, ma esistono altre possibilità come il minor
consumo d'energia e traiettorie con particolari proprietà geometriche, come la minor curvatura.


% Chapter 3

\chapter{Raceline Optimization}
\label{chap:opt}

% Chapter 4
% ======================================================================
% col: 20

\chapter{Implementazione}
In questo capitolo verranno descritti i passaggi implementati per generare la raceline ottima data una
mappa.

Il lavoro svolto è stato basato sulla seguente repository GitHub:
\url{https://github.com/CL2-UWaterloo/Raceline-Optimization}.

\bigskip
\noindent I passaggi ad alto livello sono i seguenti:
\begin{enumerate}
	\item Ottenere l'immagine di una mappa (per esempio attraverso \hyperref[par:slam]{SLAM});
	\item Eventualmente ripurla di imperfezioni e ottenere un circuito chiuso; 
	\item Generare la centerline e calcolare la larghezza del circuito;
	\item Applicare gli algoritmi di generazione della raceline;
\end{enumerate}

%TODO:spiegare occupancy grid?
La mappa digitalizzata in file immagine che rappresenta una occupancy grid, mentre
i percorsi, che siano raceline o centerline, sono in formato csv con almeno due colonne rappresentanti la
posizione $(x,y)$ di ogni sample e almeno una terza colonna per la velocità per quanto riguarda la
raceline. Successivamente si è scelto di esportare anche l'orientamento del robot rispetto alla mappa e
l'accelerazione per ogni sample.

\section{Generazione centeline}
Affinchè l'algoritmo per l'estrazione della linea di riferimento restituisca un risultato corretto, come
detto in precedenza, l'immagine della mappa deve rappresentare un circuito chiuso e con bordi ben
definiti e lisci, come effettivamente sarebbe un circuito di F1.
Dunque se si acquisice una mappa con SLAM come in figura \ref{fig:slam_map} è necessario modificarla come
in figura \ref{fig:slam_map_mod}. Durante l'implentazione di questa tesi si è deciso di usare le mappe
dei circuiti da gara più famosi forniti da F1TENTH stesso \cite{f1tenth-gitmaps}.

\begin{figure}[H]
	\begin{minipage}[c]{0.47\textwidth}
		\includegraphics[width=1\textwidth]{slam_map.png}
		\caption{\raggedright Mappa risultante da SLAM}
		\label{fig:slam_map}
	\end{minipage}\hfill
	\begin{minipage}[c]{0.47\textwidth}
		\includegraphics[width=1\textwidth]{slam_map_mod.png}
		\caption{Modifica dell'immagine \ref{fig:slam_map} per creare un circuito chiuso}
		\label{fig:slam_map_mod}
	\end{minipage}
\end{figure}

Operativamente, l'algoritmo usato per la generazione viene dal mondo dell'elaborazione delle immagini e
si chiama EDT (Euclidian Distance Transform), che calcola la distanza euclidea per ogni pixel
dell'immagine dal background: in questo caso, lo sfondo preso in considerazione sono i punti non
esplorati, ovvero quelli esterni ai bordi del circuito.
Un esempio di applicazione di EDT si trova all'immagine \ref{fig:edt-ex}.

% \begin{figure}[h]
% 	\begin{minipage}[c]{1\textwidth}
% 		\includegraphics[width=1\textwidth]{edt-ex.png}
% 		\caption{Esempio di applicazione dell'agortimo EDT su una immagine binaria, l'immagine a destra
% 		mostra il risultato indicando la distanza euclidea dal background per ogni pixel}
% 		\label{fig:edt-ex}
% 	\end{minipage}\hfill
% \end{figure}

Un aspetto da sottolineare è che le immagini fornite da F1TENTH rappresentano una occupancy grid
ternaria, in cui il grigio corrisponde alle zone non esplorate; per sfruttare l'algoritmo sopra citato è
necessario che i pixel grigi vengano convertiti in nero, perchè  è il valore considerato come background
dall'algoritmo EDT. Prendendo come riferimento il circuito di Monza all'immagine \ref{fig:monza-binary}, il
risultato dell'algoritmo si può vedere all'immagine \ref{fig:monza-edt}.

\begin{figure}[H]
	\begin{minipage}[c]{1\textwidth}
		\includegraphics[width=1\textwidth]{edt-ex.png}
		\caption{Esempio di applicazione dell'agortimo EDT su una immagine binaria, l'immagine a destra
		mostra il risultato indicando la distanza euclidea dal background per ogni pixel}
		\label{fig:edt-ex}
	\end{minipage}\hfill
	\vspace{1cm}
	\begin{minipage}[c]{0.47\textwidth}
		\includegraphics[width=1\textwidth]{monza-binary.png}
		\caption{Immagine della occupancy grid binaria del circuito di Monza}
		\label{fig:monza-binary}
	\end{minipage}\hfill
	\begin{minipage}[c]{0.47\textwidth}
		\includegraphics[width=1\textwidth]{monza-edt.png}
		\caption{\raggedright Risultato dell'algoritmo EDT mostrato sulla curva 1-2 del circuito di Monza}
		\label{fig:monza-edt}
	\end{minipage}
\end{figure}

Il passo successivo è quello di ottenere solo la parte centrale, ovvero quella con i valori di bianco
più alti e ridurla a un singolo pixel. In Python, questa operazione, viene eseguita dalla funzione
\verb|skeletonize()| del pacchetto \verb|scikit-image|. Seguendo l'esempio con il circuito di Monza, il
risultato di questa fase si può osservare all'immagine \ref{fig:monza-skel}.

A questo punto è necessario campionare il percorso di un pixel trovato: partendo da un punto del
percorso, si applica una ricerca DFS (Depth First Search) per trovare i successivi pixel bianchi. Dunque
da un pixel del percorso, si cerca il primo pixel bianco in tutte le direzioni che non sia già stato
esplorato: seguendo questo processo si ottengono le posizioni dei pixel e la loro distanza dai bordi, si
trasformano nel frame della mappa e esportano queste informazioni in un csv contentente le colonne
\verb|x, y, width_left, width_right|.

\begin{figure}[h]
	\begin{center}
		\includegraphics[width=0.3\textheight]{monza-skel.png}
	\end{center}
	\caption{Visualizzazione della centerline di Monza, il punto verde indica il punto di partenza}\label{fig:monza-skel}
\end{figure}


\section{Esecuzione dell'ottimizzazione}
Ad alto livello, l'ottimizzazione cerca iterativamente di aggiustare la centerline in modo tale da
ottenere il risultato migliore, come spiegato nel capitolo \ref{chap:opt} (pag. \pageref{chap:opt}).
Questa fase non richiede solamente di eseguire il programma solutore, ma è necessario innanzitutto
configurare i parametri che l'algortimo usa. Questo compito si chiama \textit{tuning} dei parametri
e consiste nel trovare quelli che meglio si adattano nel trovare la soluzione migliore.

Durante lo studio di questa tesi il principale parametro modificato durante le prove è stato lo
\textit{stepsize}, ovvero la distanza, in metri, tra un sample e l'altro. In particolare si distinguono
tre stepsize diversi:
\begin{itemize}
	\item \verb|stepsize_prep|: viene usato per l'interpolazione lineare prima dell'approssimazione del
		percorso con la spline;
	\item \verb|stepsize_reg|: usato durante l'ottimizzazione per l'interpolazione della spline;
	\item \verb|stepsize_interp_after_opt|: usato per campionare la spline dopo l'ottimizzaione e da esportare in csv;
\end{itemize}

\noindent Altri parametri fanno riferimento alla dinamica del veicolo, come accelerazione massima e velocità
massima, e grandezze fisiche del veicolo, come massa, lunghezza e larghezza compresa on un margine di
sicurezza aggiuntivo.


% Chapter 5
% ======================================================================
% col: 20

\chapter{Analisi dei risultati}
% In questo capitolo vengono presentati i risultati 

\section{Metriche}
Le metriche usate in questo studio sono le seguenti:
\begin{itemize}
	\item Laptime 
	\item Lunghezza tracciato 
	\item Velocità media, mediana, minima e la sua deviazione standard
	\item Accelerazione mediana, massima e minima
	\item Curvatura mediana e la sua deviazione standard
	\item Tempo di esecuzione dei tre algoritmi
\end{itemize}
Le metriche scelte permettono di confrontare le tre strategie considerando quale criterio 
ottimizzano.

\section{Grafici}

% Chapter 6
\chapter{Conclusioni}

Di seguito si discute dei risultati ottenuti durante l'analisi ed eventuali sviluppi futuri.

\section{Risultati ottenuti}
Lo studio di questa tesi si è concentrato sulle soluzioni trovate in letteratura per l'ottimizzazione del
\textit{percorso globale} data la mappa di un circuito. Nello specifico, sono state analizzate \textit{tre strategie} di
ottimizzazione: il percorso più breve, il percorso con la curvatura minima e il percorso con il lap time
minore. Le tre strategie vengono modellate secondo un problema di controllo ottimale.

Dalle analisi svolte e da sperimentazioni svolte con il simulatore è emerso che la formulazione delle
prime due strategie risulta troppo semplice per poter essere applicata anche in un simulatore: nel caso
della curvatura minima, sebbene sia l'obiettivo della strategia, la velocità mediana è troppo alta -- nel
caso in cui la massima sia impostata in egual modo per tutte le tre strategie -- mentre per il percorso
con lunghezza minima la traiettoria risultante è troppo vicina ai bordi e perciò difficile da far
seguire ad un controller. I risultati migliori sono stati ottenuti dalla strategia del tempo minimo,
principalmente perchè integra i coefficienti d'attrito tra gli pneumatici e la strada, dunque risulta un
modello più realistico, soprattutto nel caso come questo, dove si spinge il veicol ai suoi limiti.


\section{Sviluppi futuri}
Un possibile miglioramento è sicuramente quello di raffinare meglio i parametri degli algoritmi, in modo
tale che producano traiettorie praticabili: dunque diminuire la velocità massima per la strategia
mincurv, o comunque trovare una relazione tra quella massima delle altre due e quella di mincurv, e
aumentare il distanziamento tra il bordo del circuito e la traiettoria così da permettere il passaggio
del veicolo. Inoltre, si potrebbe ulteriormente aggiornare la modellazione del problema di QP sottostante
inserendo nuovi vincoli.

Un ulteriore approccio alla risoluzione del problema è quello di implementare e analizzare algoritmi di
ottimizzazione basati su \textit{strategie evolutive}, come il già citato CMA-ES (cap.~\ref{chap:opt}
pag.~\pageref{chap:opt}). \cite{lection22} In generale, è possibile esplorare anche altri algoritmi di
ottimizzazione matematica.

\bigskip
\noindent Un problema comune in questi contesti è quello che viene chiamato \textit{Sim2Real}, ovvero quello di \textit{trasportare}
il lavoro svolto dal simulatore, o comunque da un modello teorico, alla realtà. Il modello è una
semplificazione della realtà, e in alcuni casi questo può portare ad un comportamento errato se portato
direttamente in hardward. Il passaggio da simulatore a realtà spesso è seguito da un ulteriore fase di
tuning dei parametri degli algoritmi così da adattarli al meglio alla complessità della realtà. Un
possibile sviluppo futuro, dunque, potrebbe essere quello di costruire il robot e implementare il lavoro
svolto in simulazione, testarlo ed adattarlo.


\raggedright
% thanks to https://web.archive.org/web/20241003144042/https://old.reddit.com/r/LaTeX/comments/1bdqmf0/include_bibliography_in_toc_and_change_name_of/kuour1c/
\printbibliography[heading=bibintoc]
\end{document}
