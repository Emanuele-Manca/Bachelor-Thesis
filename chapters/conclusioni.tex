\chapter{Conclusioni}

Di seguito si discute dei risultati ottenuti durante l'analisi ed eventuali sviluppi futuri.

\section{Risultati ottenuti}
Lo studio di questa tesi si è concentrato sulle soluzioni trovate in letteratura per l'ottimizzazione del
\textit{percorso globale} data la mappa di un circuito. Nello specifico, sono state analizzate \textit{tre strategie} di
ottimizzazione: il percorso più breve, il percorso con la curvatura minima e il percorso con il lap time
minore. Le tre strategie vengono modellate secondo un problema di controllo ottimale.

Dalle analisi svolte e da sperimentazioni svolte con il simulatore è emerso che la formulazione delle
prime due strategie risulta troppo semplice per poter essere applicata anche in un simulatore: nel caso
della curvatura minima, sebbene sia l'obiettivo della strategia, la velocità mediana è troppo alta -- nel
caso in cui la massima sia impostata in egual modo per tutte le tre strategie -- mentre per il percorso
con lunghezza minima la traiettoria risultante è troppo vicina ai bordi e perciò difficile da far
seguire ad un controller. I risultati migliori sono stati ottenuti dalla strategia del tempo minimo,
principalmente perchè integra i coefficienti d'attrito tra gli pneumatici e la strada, dunque risulta un
modello più realistico, soprattutto nel caso come questo, dove si spinge il veicol ai suoi limiti.


\section{Sviluppi futuri}
Un possibile miglioramento è sicuramente quello di raffinare meglio i parametri degli algoritmi, in modo
tale che producano traiettorie praticabili: dunque diminuire la velocità massima per la strategia
mincurv, o comunque trovare una relazione tra quella massima delle altre due e quella di mincurv, e
aumentare il distanziamento tra il bordo del circuito e la traiettoria così da permettere il passaggio
del veicolo. Inoltre, si potrebbe ulteriormente aggiornare la modellazione del problema di QP sottostante
inserendo nuovi vincoli.

Un ulteriore approccio alla risoluzione del problema è quello di implementare e analizzare algoritmi di
ottimizzazione basati su \textit{strategie evolutive}, come il già citato CMA-ES (cap.~\ref{chap:opt}
pag.~\pageref{chap:opt}). \cite{lection22} In generale, è possibile esplorare anche altri algoritmi di
ottimizzazione matematica.

\bigskip
\noindent Un problema comune in questi contesti è quello che viene chiamato \textit{Sim2Real}, ovvero
quello di \textit{trasportare} il lavoro svolto dal simulatore, o comunque da un modello teorico, alla
realtà. Il modello è una semplificazione della realtà, e in alcuni casi questo può portare ad un
comportamento errato se portato direttamente in hardward. Il passaggio da simulatore a realtà spesso è
seguito da un ulteriore fase di tuning dei parametri degli algoritmi così da adattarli al meglio alla
complessità della realtà. Un possibile sviluppo futuro, dunque, potrebbe essere quello di costruire il
robot e implementare il lavoro svolto in simulazione, testarlo ed adattarlo.
