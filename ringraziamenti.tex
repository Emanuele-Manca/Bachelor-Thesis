
\chapter*{Ringraziamenti}
\label{chap:ringr}
\addcontentsline{toc}{chapter}{\nameref{chap:ringr}}

Seppur banale, come prima cosa vorrei ringraziare \textbf{i miei genitori} che mi hanno permesso di
intraprendere questo percorso. Li ringrazio per avermi dato la libertà di seguire le mie passioni e, nei
momenti giusti, costringermi a valutare anche altre opzioni: senza \textit{mia mamma} non avrei
conosciuto l'informatica.

Sebbene non abbiamo avuto un ruolo rilevante durante il percorso universitario in senso stretto, vorrei
ringraziare anche i miei secondi genitori, \textbf{i nonni materni}, con cui sono cresciuto: grazie
\textit{nonna} e grazie \textit{nonno} per tutte le cose che avete fatto per me in tutti questi anni.

Durante questa esperienza ho incontrato i miei attuali amici e altri mi hanno accompagnato fin da prima;
vorrei ringraziarli tutti per il supporto che mi hanno dato durante questi anni, tra esami, appunti e ore
passate su Discord a studiare, chiaccherare e giocare assieme. Con alcuni di loro ho fatto anche nuove
esperienze tra vacanze e gite e conosciuto altra gente: insomma, chi di più chi di meno, mi hanno aiutato
a crescere e maturare. 

Mi accompagnano fin dalle superiori \textbf{Vincenzo Siano} (\textit{WinCE} detto vince) e
\textbf{Cristian Pozzi} (\textit{criii}/\textit{pohtzie} detto pozzi), senza di loro sarebbe stato più
difficile iniziare l'università; ma abbiamo fin da subito conosciuto -- non in ordine di importanza --
\textbf{Federico Marcelli Fabiani} (\textit{fede}/\textit{fefè}), \textbf{Davide Papasodaro}
(\textit{papass}/\textit{dadæ}), \textbf{Gabriele Giorgio} (\textit{gabri}), \textbf{Marco Morandi}
(\textit{marco}/\textit{mora}), \textbf{Gabriele Gilberti} (\textit{gigi}), \textbf{Federico Coscia}
(\textit{fedoscia}).

È stato un percorso travagliato e pieno di ostacoli, tra esami, scadenze e {\footnotesize(alcuni)}
professori; ho cercato di aiutare a mia volta coloro che l'hanno fatto con me, ciò nonostante arriveremo
tutti al nostro traguardo finale {\footnotesize (chi prima, chi dopo)}. 
\vskip 2cm
\raggedleft \textit{È stato divertente, ragazzi.}
