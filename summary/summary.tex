\documentclass[12pt,a4paper]{article}

% =============== PACKAGES =================
% \usepackage{hyperref} % per i link e riferimenti
\usepackage[italian]{babel} % utile per suddividere le parole quando si va a capo
\usepackage{geometry} % geometria della pagina
\usepackage[a-1b]{pdfx}

% =============== GENERAL SETTINGS =========
\geometry{a4paper,top=2cm,bottom=2cm,left=2cm,right=2cm}
% \renewcommand{\baselinestretch}{1.3} % intelinea

\title{\textbf{Strategie di Raceline Optimization\\in F1TENTH Autonomous Racing}}
\author{Emanuele Manca, Matr. 978785}
\date{}

% =============== DOCUMENT =================
\begin{document}
\maketitle
\noindent L'argomento di tesi trattato è l'ottimizzazione della traiettoria di gara nel contesto di gare
automobilistiche di robot, nello specifico caso di F1TENTH; quest'ultima è una community di appassionati
e ricercatori nell'ambito della robotica che promuove lo sviluppo di robot autonomi e organizza gare
internazionali di robot da corsa.

La tesi si concentra sul \textit{pianificare} un percorso ottimale -- secondo criteri scelti -- che
prende in considerazione l'intero circuito; la traiettoria generata deve rimanere nei limiti fisici del
veicolo ed evitare che lo stesso vada a scontrarsi con i bordi del circuito. L'auto a guida autonoma
utilizzerà questa traiettoria come riferimento poi durante la gara, al fine di completare un giro. Il suo
obiettivo sarà quindi cercare di attuare la traiettoria sviluppata durante questo lavoro.

\bigskip
\noindent L'obiettivo della tesi è stato lo studio e l'analisi di tre diversi criteri di ottimalità per
la generazione di un percorso pianificato: il percorso \textit{più breve}, il percorso con la
\textit{curvatura minima} e il percorso col \textit{minor tempo} sul giro.

Inizialmente è stato effettuato uno studio di ROS, delle nozioni significative alla base della guida
autonoma, della piattaforma di F1TENTH e, infine, delle soluzioni in letteratura per il global
planning.

Prima della sperimentazione è stata necessaria una configurazione dell'ambiente di sviluppo dove è stata
implementata, utilizzando delle configurazioni e delle librerie fornite dalla community F1TENTH,
l'infrastruttura necessaria nel sviluppare un veicolo per l'autonomous racing funzionante e comprensiva
di ogni suo modulo. I metodi per il planning di traiettorie ottimali studiati durante questo lavoro di
tesi sono stati sviluppati generando degli input specifici per i circuiti scelti -- \textit{le
centerline} -- con l'uso di tecniche per l'elaborazione di immagini. La centerline è quel tracciato,
equidistante dai bordi del circuito, da cui il risolutore inizia l'ottimizzazione. 

A questo punto, sono state effettuate sperimentazioni per raffinare i parametri degli algoritmi coinvolti
(\textit{tuning}) e sono state apportate modifiche al codice per lo scopo della tesi e per una migliore
organizzazione.

Dopo di che sono state raccolte diverse raceline per ogni strategia di ottimizzazione e, con l'uso di
Jupyter Notebook, sono state confrontate e analizzate tra loro secondo alcune metriche (es: lap time,
velocità, lunghezza). Infine, sono stati generati grafici e immagini delle raceline e delle loro proprietà
sul circuito.

\bigskip
\noindent Dall'analisi è emerso come la modellazione matematica dei problemi in programmazione quadratica
della strategia del percorso minimo e della minima curvatura produca risultati difficilmente applicabili
ad un veicolo robotico; potrebbero essere migliorati applicando alcune variazioni ai parametri e
integrando ulteriori vincoli sulla dinamica del veicolo. Mentre la strategia del tempo minimo è risultata
più realistica perchè integra nella modellazione l'attrito degli pneumatici con il terreno.
\end{document}
